\documentclass[a4paper]{article}

\usepackage[english]{babel}
\usepackage[utf8]{inputenc}
\usepackage{amsmath}
\usepackage{amsfonts}
\usepackage{graphicx}
\usepackage[dvipsnames]{xcolor}
\usepackage[colorinlistoftodos]{todonotes}
\usepackage{algorithm}
\usepackage{geometry}
\usepackage{algpseudocode}
\usepackage{amsmath}

\geometry{
    paperwidth=220mm,
    % paperheight=16383pt,
    paperheight=1500pt,
    left=20mm,
    right=20mm,
    top=20mm,
    bottom=20mm,
}
\setcounter{totalnumber}{100}

\DeclareMathOperator*{\argmax}{argmax}
\DeclareMathOperator*{\aggregate}{aggregate}

\algnewcommand{\Initialize}[1]{%
          \State \textbf{Initialize:}
          \Statex \hspace*{\algorithmicindent}\parbox[t]{.8\linewidth}{\raggedright #1}
        }

\begin{document}
\begin{algorithm}
\caption{Utilities}
    \begin{algorithmic}[1]
      \Function{ProposeMessage}{$step$, $QC$}
        \State return  {Message}(
          \State \indent \textbf{Step}: step,
          \State \indent \textbf{Height}: self.height,
          \State \indent \textbf{Round}: self.round,
          \State \indent \textbf{Block}: self.block,
          \State \indent \textbf{JustifyQC}: QC
          \State )
      \EndFunction
      \vspace {5 mm}
      \Function{VoteMessage}{$step$, $block$}
        \State partialSignature = sign(step, self.height, self.round, block)
        \State message = {Message}(
          \State \indent \textbf{Step}: step,
          \State \indent \textbf{Height}: self.height,
          \State \indent \textbf{Round}: self.round,
          \State \indent \textbf{Signature}: partialSignature,
          \State )
        \State return message
      \EndFunction
      \vspace {5 mm}
      \Function{InterruptRound}{}
        \State self.round = self.round + 1
        \State self.step = PREPARE
        \State self.leader = nil
        \newline {\color{Red} \hspace{4 mm} // IMPORTANT: Do we really need a TimeoutQC here? What would it be?}
        \State \textbf{broadcast} ProposeMessage(NEWROUND, self.highPrepareQC) → valSet
      \EndFunction
      \vspace {5 mm}
      \Function{NewHeight}{}
        \State self.height = height+1
        \State self.round = 0
        \State self.step = PREPARE;
        \State {\color{MidnightBlue} self.highPrepareQC} = nil
        \State {\color{Orange} self.LockedQC} = nil
        \State self.leader = nil
        \State self.valSet = CommitBlock() {\color{Red} // IMPORTANT: Why are we updating valSet here?}
        \newline {\color{Red} \hspace{4 mm} // IMPORTANT: Do we really need a CommitQC here? What would it be?}
        \State \textbf{broadcast} ProposeMessage(NEWROUND, nil) → valSet
      \EndFunction
      \vspace{5 mm}
      \Function{IsValidProposal}{$prepareMsg$}
         \If{$IsValidQC(prepareMsg)$}
            \If{
                {\color{Orange} $self.lockedQC$}$ == nil$
                \State $\lor$ {\color{Orange} $self.lockedQC.height$}$ < prepareMsg.JustifyQC.height$
                \State $\lor$ {\color{Orange} $self.lockedQC.round$}$ < prepareMsg.JustifyQC.round$
            } {\color{Gray} { // liveness rule}}
             \State \textbf{return} True
           \Else
             \If {$prepareMsg.JustifyQC.block$ extends from {\color{Orange} $self.lockedQC$}$.block$} {\color{Gray} { // safety rule}}
                \State\textbf{return} True
             \EndIf
           \EndIf
         \EndIf
         \State \textbf{return} False
       \EndFunction
    \end{algorithmic}
\end{algorithm}
\begin{algorithm}
  \caption{HotPOKT}x
  \begin{algorithmic}[1]
      \Initialize{
        \strut$height=1$ {\color{Gray} // Special case for Genesis (height = 0) omitted} \\
        \strut$round=0$ \\
        \strut$step=NEWROUND$ \\
        \strut$block=nil$ { \color{Gray} // The block being proposed by the leader} \\
        {\color{MidnightBlue} \strut$highPrepareQC$}$=nil$ \\
        {\color{Orange} \strut$lockedQC$}$=nil$ \\
        \strut$leader=nil$ \\
        \strut$valSet=UniversalValidatorSet$ \\
      }
      \bigskip
      \State \emph{ \color{Gray} // New rounds repeated for each height until a ProposalBlock is successfully committed.}
      \For{round ← true }
        \vspace {5 mm}
        \newline ${\color{Bittersweet} {\rhd Leader Election}}$
        \State leader = SelectLeader(height, round, valSet, ...)
        \State isLeader = (self.Addr == leader.Addr)
        \State step = isLeader ? NEWROUND : PREPARE
        \vspace {5 mm}
        \newline ${\color{Bittersweet} {\rhd Prepare Phase}}$
        \If{$IsLeader$}
            \newline {\color{Red} \hspace{10mm} // IMPORTANT: Need to validate each prevPrepareQC, discard invalid ones, and see if we still have {+2/3}}
            \State newRoundMsgs = \textbf{wait} for {+2/3} ProposeMessages\{\textsc{NEWROUND}, height, round, {\color{MidnightBlue} prevPrepareQC}\} ← valSet
            \State step = PREPARE
            \State {\color{MidnightBlue} prevHighPrepareQC} = ${\argmax_{m \in newRoundMsgs} \{m.JustifyQC.Height\}.JustifyQC}$ {\color{Gray} { // may be nil}}
            \If{
                {\color{MidnightBlue} $prevHighPrepareQC$}$ == nil$
                \State $\lor$ {\color{MidnightBlue} $prevHighPrepareQC$}$.height < height$
                \State $\lor$ {\color{MidnightBlue} $prevHighPrepareQC$}$.round < round$
            }
                \State block = NewProposalBlock() {\color{Gray} // extends from highPrepareQC.block if not nil}
            \Else
                \newline {\color{Red} \hspace{10mm} // IMPORTANT: Why are we not checking highPrepareQC or LockedQC here?}
                \State block = {\color{MidnightBlue} prevHighPrepareQC}.block
            \EndIf
            \newline {\color{Red} \hspace{10mm} // IMPORTANT: No QC necessary if prevHighPrepareQC is nil?}
            \State \textbf{broadcast} ProposeMessage(PREPARE, {\color{MidnightBlue} prevHighPrepareQC}) → valSet
        \Else \quad {\color{Gray} { // IsReplica}}
            \State prepareMsg = \textbf{wait} for ProposeMessage\{PREPARE, height, round, block, prepareQC\} ← leader
            \If{$IsValidProposal(prepareMsg)$}
              \newline {\color{red} \hspace{15 mm} // IMPORTANT: If proposal is valid, is it safe to unlock here?}
              \State {\color{Orange} lockedQC} = nil
              \State {\color{MidnightBlue} highPrepareQC} = nil
              \State step = PRECOMMIT
              \State \textbf{send} VoteMsg(PREPARE, prepareMsg.block) → leader
            \Else
              \State InterruptRound()
            \EndIf
          \EndIf
        \bigskip
        \newline ${\color{Bittersweet} {\rhd Precommit Phase}}$
        \If{$IsLeader$}
            \State prepareMsgs = \textbf{wait} for {+2/3} VoteMessage\{\textsc{PREPARE}, height, round, partialSig\} ← valSet
            \State step = PRECOMMIT
            \newline {\color{Red} \hspace{10 mm } // IMPORTANT: Validate each partialSig and discard}
            \State thresholdSig = ${\aggregate_{m \in prepareMsgs} \{m.partialSignature\}}$
            \State prepareQC = QC{step, height, block, thresholdSig}
            \State \textbf{broadcast} ProposeMessage(PRECOMMIT, height, round, prepareQC) → valSet
        \Else \quad {\color{Gray} { // IsReplica}}
          \State preCommitMsg = \textbf{wait} for ProposeMessage\{PRECOMMIT, height, round, prepareQC\} ← leader
          \If{$IsValidQC(preCommitMsg.JustifyQC)$}
            \State {\color{MidnightBlue}highPrepareQC} = preCommitMsg.JustifyQC
            \State step = COMMIT
            \State \textbf{send} VoteMessage(PRECOMMIT, preCommitMsg.block) → leader
          \Else
            \State InterruptRound()
          \EndIf
        \EndIf
      \vspace {5 mm}
      \newline ${\color{Bittersweet} {\rhd Commit Phase}}$
      \If{$IsLeader$}
          \State preCommitMsgs = \textbf{wait} for {+2/3} VoteMessage\{\textsc{PRECOMMIT}, height, round, partialSig\} ← valSet
          \State step = COMMIT
          \newline {\color{Red} \hspace{10 mm } // IMPORTANT: Validate each partialSig and discard}
          \State thresholdSig = ${\aggregate_{m \in preCommitMsgs} \{m.partialSignature\}}$
          \State preCommitQC = QC\{step, height, block, thresholdSig\}
          \State \textbf{broadcast} ProposeMessage(COMMIT, height, round, preCommitQC) → valSet
        \Else \quad {\color{Gray} { // IsReplica}}
          \State commitMsg = \textbf{wait} for ProposeMessage\{COMMIT, height, round, preCommitQC\} ← leader
          \State step = COMMIT
          \If{$IsValidQC(commitMsg)$}
            \State {\color{Orange} lockedQC} = commitMsg.precommitQC
            \State \textbf{send} VoteMessage(COMMIT, commitMsg.block) → leader
          \Else
            \State InterruptRound()
          \EndIf
        \EndIf
        \State
        \newline ${\color{Bittersweet} {\rhd Decide Phase}}$
        \If{$IsLeader$}
          \State commitMsgs = \textbf{wait} for {+2/3} VoteMessage\{\textsc{COMMIT}, height, round, partialSig\} ← valSet
          \State step = DECIDE
          \State thresholdSig = ${\aggregate_{m \in commitMsgs} \{m.partialSignature\}}$
          \State commitQC = QC\{step, height, block, thresholdSig\}
          \State \textbf{broadcast} ProposeMessage(DECIDE, height, round, commitQC) → valSet
          \newline {\color{Red} \hspace{10 mm} // IMPORTANT: The leader also needs to process the message like a replica to commit the block and start a new height.}
        \Else \quad {\color{Gray} { // IsReplica}}
            \State decideMsg = \textbf{wait} for ProposeMessage\{DECIDE, height, round, commitQC\} ← leader
            \State step = DECIDE
            \If{$IsValidQC(decideMsg)$}
                \newline {\color{red} \hspace{15 mm} // COMMIT BLOCK}
                \State NewHeight() {\color{Red} // IMPORTANT: Do we need to pass commitQC here?}
            \Else
                \State InterruptRound()
          \EndIf
        \EndIf
      \EndFor
      \State
  \end{algorithmic}
\end{algorithm}
\end{document}
